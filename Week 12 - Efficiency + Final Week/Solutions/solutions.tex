\documentclass[12pt]{article}

% Packages
\usepackage{lipsum} % For dummy text
\usepackage{graphicx} % For including images
\usepackage{amsmath} % For mathematical symbols and equations
\usepackage[margin=1in]{geometry} % Decrease margins
\usepackage{forest}
\usepackage{titlesec}
\usepackage{tikz}
\usepackage{pgfplots} % Add the pgfplots package for plotting

% Title and author
\title{Solutions to Week 12 Exercises}
\author{Ibrahim Chehab}
\usepackage{amsthm} % Add the amsthm package to define the proof environment
\usepackage{xcolor}

\usepackage{amsmath} % For mathematical symbols and equations

% Ceil command
\newcommand{\ceil}[1]{\lceil #1 \rceil}


\definecolor{codegreen}{rgb}{0,0.6,0}
\definecolor{codegray}{rgb}{0.5,0.5,0.5}
\definecolor{codepurple}{rgb}{0.58,0,0.82}
\definecolor{backcolour}{rgb}{0.95,0.95,0.92}

\usepackage{listings}
\lstdefinestyle{mystyle}{
    commentstyle=\color{codegreen},
    keywordstyle=\color{blue},
    stringstyle=\color{codepurple},
    basicstyle=\ttfamily\small,
    breakatwhitespace=false,
    breaklines=true,
    captionpos=b,
    keepspaces=true,
    showspaces=false,
    showstringspaces=false,
    showtabs=false,
    tabsize=2
}


\begin{document}

\maketitle

\section{Questions from FSG Slides}

\subsection{The Tightest Bound} 
Peace and d.aki are arguing over the tightest upper-bound for the following function. Peace argues that it's $\mathcal{O}(n^2)$ while d.aki argues that it's $\mathcal{O}(n^3)$. Given the following function, who is correct? Justify your answer.

\subsubsection{}

\begin{lstlisting}[language=Python,style=mystyle]
def mystery(n: int):
L = []
for i in range(n):
    for j in range(min(50, n)):
        L.insert(0, j)
\end{lstlisting}

\textbf{Solution:}
To solve this question, we will first begin by analyzing the time complexity of the inner loop. The inner loop runs for $\min(50, n)$ iterations. This means that the inner loop runs for $50$ iterations if $n \geq 50$ and $n$ iterations if $n < 50$.

Inside this loop, we have an $\mathcal{O}(n)$ operation, which is the \texttt{insert} operation. Then, encapsulating all of this, we have an outer loop that runs for $n$ iterations.

With this information, we will construct a piecewise function $T(n)$ that represents the time complexity of the function \texttt{mystery}.

\begin{equation*}
    T(n) = \begin{cases}
        50n^2 & \text{if } n \geq 50 \\
        n^3 & \text{if } n < 50
    \end{cases}
\end{equation*}

Notice, for sufficiently large $n$ (i.e. $n \geq 50$), the complexity of the second loop drops out. This can be seen when graphing the function \texttt{min(50, n)}.

\begin{center}
    \begin{tikzpicture}
        \begin{axis}[
            axis lines = left,
            xlabel = $n$,
            ylabel = {$\min(50, n)$},
            ymin=0, ymax=55, % Adjust y-axis limits
            xmin=0, xmax=100, % Adjust x-axis limits to show behavior for n > 50
            legend pos=north west,
        ]
        % Plot for n when n <= 50
        \addplot [
            domain=0:50, 
            samples=100, 
            color=red,
        ]
        {x};
        \addlegendentry{$n \leq 50$}
        
        % Plot for n when n > 50
        \addplot [
            domain=50:100, 
            samples=100, 
            color=red,
        ]
        {50};
        \addlegendentry{$n > 50$}
        
        \end{axis}
        \end{tikzpicture}
\end{center}
Notice, for sufficiently large $n$, the inner-most loop runs for 50 iterations, making it run in constant time (i.e: regardless of the value of $n$, the inner loop runs for 50 iterations). 

This means that for sufficiently large $n$, the time complexity of the function \texttt{mystery} is $\mathcal{O}(50n^2) \implies \mathcal{O}(n^2)$. Therefore, Peace is correct.

With that said, however, d.aki is still correct to some degree. We know that for our function $T$ to be an element of $\mathcal{O}(n^3)$, there must exist some $c > 0$ and $n_0 > 0$ such that $T(n) \leq cn^3$ for all $n \geq n_0$. Since our function $T \in \mathcal{O}(n^2)$, it is vacuously true that $T(n) \leq cn^3$ for all $n \geq n_0$ for some $c > 0$ and $n_0 > 0$. Therefore, d.aki is correct in that the function is also $\mathcal{O}(n^3)$. However, this is \textbf{not} the tightest upper-bound for the function.

Therefore, while both are correct, Peace is more correct in this case.

\subsubsection{}
\begin{lstlisting}[language=Python,style=mystyle]
def mystery2(n: int):
  L = []
  for i in range(n):
      for j in range(max(50, n)):
          L.insert(0, j)
\end{lstlisting}

\textbf{Solution:}
As the question suggests, we will modify our piecewise $T(n)$ function to account for the change in the inner loop. The inner loop now runs for $\max(50, n)$ iterations. This means that the inner loop runs for $50$ iterations if $n < 50$ and $n$ iterations if $n \geq 50$.

This translates to the following
\begin{equation}
    T(n) = \begin{cases}
        50n^2 & \text{if } n < 50 \\
        n^3 & \text{if } n \geq 50
    \end{cases}
\end{equation}

Notice that now for sufficiently large $n$, the time complexity of the function \texttt{mystery2} is $\mathcal{O}(n^3)$. Therefore, d.aki is correct in this case.

Further, unlike the previous function, the function \texttt{mystery2} is not $\mathcal{O}(n^2)$. This is because $\mathcal{O}(n^3) \not\subseteq \mathcal{O}(n^2)$. Therefore, Peace is incorrect in this case.

\subsection{FindMiiComplexity}
What is the worst-case asymptotic complexity of the following function? Justify your answer.
\begin{lstlisting}[language=Python,style=mystyle]
def foo(x: int):
    n = x
    for i in range(50):
    if n <= 50:
        for j in range(n):
        x += 1
    if n <= 100:
        for j in range(n):
        if n <= 75:
            for k in range(n):
            x += 1
\end{lstlisting}

\textbf{Solution:}

We will follow a similar approach to the previous questions. We will analyze the time complexity of the inner loops and construct a piecewise function $T(n)$ that represents the time complexity of the function \texttt{foo}.

The outermost loop runs exactly $50$ times, as it is constant and not tied to any function call or input. Creating $T_1, T_2, T_3$ (Where $T_n$ is the $nth$ loop in the function) to represent the time complexity of the inner loops, we have the following:
\begin{equation*}
    T_1(n) = 50
\end{equation*}
\begin{equation*}
    T_2(n) = \begin{cases}
        n & \text{if } n \leq 50 \\
        0 & \text{if } n > 50
    \end{cases}
\end{equation*}
\begin{equation*}
    T_3(n) = \begin{cases}
        n & \text{if } n \leq 100 \\
        0 & \text{if } n > 100
    \end{cases}
\end{equation*}
\begin{equation*}
    T_4(n) = \begin{cases}
        n & \text{if } n \leq 75 \\
        0 & \text{if } n > 75
    \end{cases}
\end{equation*}

We can see that for sufficiently large $n$, none of the outer-loops run, leaving only the outermost loop to run $50$ times. This means that the time complexity of the function \texttt{foo} is $\mathcal{O}(50) \implies \mathcal{O}(1)$. Therefore, the worst-case asymptotic complexity of the function \texttt{foo} is $\mathcal{O}(1)$.

\section*{Homework Questions}

I can't post solutions for the Exam questions I assigned since those are copyrighted and I don't feel like getting sued.

I won't post the solutions for the "unofficial" homework since it's outside of the scope of the course, and you won't see anything like it until CSC236.

\end{document}