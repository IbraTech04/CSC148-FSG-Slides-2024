\documentclass[hyperref={colorlinks,citecolor=blue,linkcolor=blue,urlcolor=blue}, aspectratio=1610]{beamer}
\usepackage{xcolor}
\usepackage{pgfpages}
\usepackage[utf8]{inputenc}
\usepackage[english]{babel}
\usepackage{amsmath}
\usepackage{amsmath,amssymb}
\usepackage{tikz}
\usepackage{forest}
\usepackage{qrcode}
\usepackage{graphicx}


\mode<presentation>
{
  \usetheme{Madrid}       % or try default, Darmstadt, Warsaw, ...
  \usecolortheme{beaver} % or try albatross, beaver, crane, ...
  \usefonttheme{structurebold}    % or try default, structurebold, ...
  \setbeamertemplate{navigation symbols}{}
  \setbeamertemplate{caption}[numbered]
} 



\definecolor{codegreen}{rgb}{0,0.6,0}
\definecolor{codegray}{rgb}{0.5,0.5,0.5}
\definecolor{codepurple}{rgb}{0.58,0,0.82}
\definecolor{backcolour}{rgb}{0.95,0.95,0.92}

\usepackage{listings}
\lstdefinestyle{mystyle}{
    commentstyle=\color{codegreen},
    keywordstyle=\color{blue},
    stringstyle=\color{codepurple},
    basicstyle=\ttfamily\small,
    breakatwhitespace=false,
    breaklines=true,
    captionpos=b,
    keepspaces=true,
    showspaces=false,
    showstringspaces=false,
    showtabs=false,
    tabsize=2
}

% \pgfpagesuselayout{resize to}[%
%   physical paper width=8in, physical paper height=6in]

\title[CSC148 MEGA FSG]{The CSC148 Mega FSG}
\author[Ibrahim and Helia]{Ibrahim Chehab and Helia Seyedmazhari}
\institute{UTM RGASC}
\date{April $9^{th}$, 2024}

\begin{document}

\begin{frame}
  \titlepage
\end{frame}

\begin{frame}{Table of Contents}
  \tableofcontents
\end{frame}

\section{Introduction}
\subsection{Welcome to the CSC148 Mega FSG!}
\begin{frame}
    \frametitle{Welcome to the CSC148 Mega FSG!}
    Welcome everyone, to the first (I think?) ever CSC148 \textit{Mega FSG!+}
    \begin{itemize}
      \item Thank you for coming! We hope you enjoy the FSG!
      \item Today we will be going over some key concepts from CSC148, and we will be doing some practice problems in preparation for the exam.
    \end{itemize}
\end{frame}

\section{Key Terms}
\begin{frame}
  \frametitle{Key Terms}
  We will now quickly go over all the key terms you need to know for the exam.\\
  \textbf{Note:}
  \begin{itemize}
    \item This is \textbf{NOT} an exhaustive or comprehensive list by any means. When in doubt, always check with a TA/Professor.
    \item We will be going backwards, here's why:
    \begin{itemize}
      \item All too often, I (Ibrahim) see people neglecting weeks 1-6 and focusing on the later weeks. This is a mistake.
      \item Like all CS courses, everything builds on top of each other. If you don't understand the basics, you won't understand the more complex stuff.
      \item We will be showing you how the content from previous weeks builds upon the more complex stuff, to hopefully emphasize the importance of understanding the basics.
    \end{itemize}
  \end{itemize}
\end{frame}

\begin{frame}
  \frametitle{Week 12 Key Terms}
  \begin{itemize} 

    \item \textbf{Big-O Notation ($\mathcal{O}(n)$)}
    \item \textbf{Big-Theta Notation ($\Theta(n)$)}
  \end{itemize}  
\end{frame}

\begin{frame}{Week 11 Key Terms}
  \begin{itemize}
    \item \textbf{Divide and Conquer} 
    \begin{itemize}
        \item \textit{\textbf{Recursion}}
    \end{itemize}
    \pause
    \item \textbf{QuickSort}
    \begin{itemize}
        \item \textit{Partitioning}
        \item \textit{Pivot}
        \item \textit{How the Pivot can affect the efficiency of the algorithm}
        \item \textit{Why does Quicksort performance vary?}
    \end{itemize}
    \pause
    \item \textbf{MergeSort}
    \begin{itemize}
      \item \textit{Merging}
      \item \textit{Why Mergesort has consistent performance}
    \end{itemize}
  \end{itemize}
\end{frame}

\begin{frame}{Week 8, 9, 10 Key Terms}
  \begin{itemize}
    \item \textbf{Root} 
    \item \textbf{Subtree} 
    \item \textbf{Branching Factor}
    \item \textbf{Height}
    \item \textbf{All Tree Traversals}
    \item \textbf{BSTs and why they're special}
    \begin{itemize}
      \item \textit{All time complexities of BST operations}
      \item \textit{How $\mathcal{O}(h)$ translates in terms of worst-case and best-case scenarios}
    \end{itemize}
  \item \textbf{Polymorphism}
  \begin{itemize}
    \item \textbf{Why?} Expression Trees abuse Polymorphism to exist
  \end{itemize}
  \end{itemize}
\end{frame}

\begin{frame}{Week 6, 7 Key Terms}
  \begin{itemize}
    \item \textbf{List Comprehension} 
    \item \textbf{Recursion} 
  \end{itemize}
\end{frame}

\begin{frame}{Week 5, 4, 3, 2, 1 Key Terms}
  \begin{itemize}
    \item \textbf{Inhertiance}
    \begin{itemize}
      \item \textit{How it relates to Polymorphism}
    \end{itemize}
    \item \textbf{Abstraction}
    \item \textbf{Stacks and Queues}
    \begin{itemize}
      \item \textit{How the implementation changes the time complexity}
      \item \textit{Why Stacks and Queues are important}
      \item \textit{\textbf{Why} they're callled \textbf{Abstract} data types}
    \end{itemize}
    \item \textbf{Linked Lists}
  \end{itemize}
\end{frame}

\section{Practice Problems}
\subsection{Did Somebody Say Palindrome?}
\begin{frame}[fragile]
  \frametitle{Practice Problem 1: Did Somebody Say Palindrome?}
  \textbf{Did Somebody Say Palindrome?}\\
  Implement a \textit{recursive} function that checks whether a given string is a palindrome.
  
  \begin{center}
    \textbf{RESTRICTIONS}:
    \begin{itemize}
      \item[i.] This function \textbf{MUST} be implemented using recursion.
      \item[ii.] This function must \textbf{NOT} mutate the original word/sentence.
      \item[iii.] You may use slicing, but you may \textbf{NOT} use the built-in reversal \texttt{[::-1]} or the \texttt{reversed()} function.
    \end{itemize}
  \end{center}
  % ansewr to the problem
  % def is_palindrome(word: str) -> bool:
  %   if len(word) <= 1:
  %       return True
  %   return word[0] == word[-1] and is_palindrome(word[1:-1])

\textit{  Here are some examples:}
  \begin{enumerate}
      \item[(a)] \texttt{ewe}
      \item[(b)] \texttt{anna}
      \item[(c)] \texttt{borrow or rob}
      \item[(d)] \texttt{taco cat}
      \item[(e)] \texttt{was it a car or a cat i saw}
      \item[(f)] \texttt{racecar}
    \end{enumerate}
    \textit{Hint: Recall Ibrahim's recursion analogy}
\end{frame}

\subsection{InsertMii}
\begin{frame}[fragile]
  \frametitle{Practice Problem 2: InsertMii}
  \textbf{InsertMii}\\
  Consider the following implementation of a Doubly Linked List:

  \begin{lstlisting}[language=Python, style=mystyle]
class DLLNode:
  """A node in a linked list."""
  item: Any
  next: Optional[DLLNode]
  prev: Optional[DLLNode]

class DoublyLinkedList:
  """A doubly linked list."""
  _first: Optional[DLLNode]
  _last: Optional[DLLNode]

  # Implementation omitted
  \end{lstlisting}

\end{frame}

\begin{frame}[fragile]
  \frametitle{Practice Problem 2: InsertMii (Cont'd)}
  Implement the following method in the \texttt{DoublyLinkedList} class:

  \begin{lstlisting}[language=Python, style=mystyle]
def insert_last(self, value: Any, after: Any) -> bool:
    """Insert a new Node with the value <value> after the LAST occurrence of the value <after> in this list.
    If <after> does not exist in the list, then do not insert anything and return False.
    The list must be correctly linked after this operation.
    >>> sl = CustomDLL([7, 2, 7, 3])
    >>> str(sl) 
    '7 2 7 3'
    >>> sl.insert_last(5, 7)
    True
    >>> str(sl)
    '7 2 7 5 3'
    >>> sl.insert_last(9, 8)
    False
    >>> str(sl)
    '7 2 7 5 3'
    """
\end{lstlisting}
\end{frame}

\subsection{The Even-Worse-Stack}
\begin{frame}[fragile]
  \frametitle{Practice Problem 3: The Even-Worse-Stack}
  Nugget has entered their \textit{evil era} and designed an evil ADT known as the \texttt{EvenWorseStack}. They've subjected Therapist to the \texttt{EvenWorseStack} and now Therapist is in a state of despair. Help Therapist by analyzing the time complexity of the \texttt{pop} method of the \texttt{EvenWorseStack} class.\\

  \begin{columns}[T,onlytextwidth]
  
    \begin{column}{0.48\textwidth}
      \begin{lstlisting}[language=Python, style=mystyle, basicstyle=\tiny]
class EvenWorseStack:
  """
  A Stack implementation designed to be slow and inefficient.
  """
  _stack: Queue
  
  def __init__(self) -> None:
      self._stack = Queue()
  
  def push(self, value: int) -> None:
      self._stack.enqueue(value)
  
  def pop(self) -> int:
      temp = Queue()
      while self._stack.size() > 1:
          temp.enqueue(self._stack.dequeue())
      value = self._stack.dequeue()
      self._stack = temp
      return value
      \end{lstlisting}
    \end{column}
    
    \begin{column}{0.48\textwidth}
      \begin{lstlisting}[language=Python, style=mystyle, basicstyle=\tiny]
class Queue:
    
    _queue: list[int]
    
    def __init__(self) -> None:
        self._queue = []
    
    def enqueue(self, value: int) -> None:
        self._queue.insert(0, value)
    
    def dequeue(self) -> int:
        index_to_remove = self.size() - 1
        value = self._queue[index_to_remove]
        self._queue = self._queue[:index_to_remove]
        return value
    
    def size(self) -> int:
        return len([i for i in self._queue])
      \end{lstlisting}
    \end{column}
    
  \end{columns}
  What is the time complexity of the \texttt{pop} method?
\end{frame}

\subsection{Efficiencii}
\begin{frame}
  \frametitle{Practice Problem 4: Efficiencii}
  \textbf{Efficiencii}\\
  Select all the statements that are \textbf{TRUE}:
  \begin{enumerate}
    \item The $n_0$ you choose does not change the final result of the efficiency class
    \item If a function is upper-bounded by $\mathcal{O}(n^2)$, it might still be $\Theta(n)$
    \item If a function is upper-bounded by $\mathcal{O}(n)$, it might still be $\Theta(n^2)$
    \item The iterative part of \texttt{QuickSort} is faster than the iterative part of \texttt{MergeSort}
    \item The recursive part of \texttt{QuickSort} is faster than the recursive part of \texttt{MergeSort}
    \item If a function is $\mathcal{O}(g(n))$, then it is also $\Theta(g(n))$
    \item If a function is $\Theta(g(n))$, then it is also $\mathcal{O}(g(n))$
\end{enumerate}
\end{frame}

\subsection{The Final Challenge\ldots}
\begin{frame}
  \frametitle{The Final Challenge\ldots}
  \textbf{Symbolab from Ohio}\\
  Peace and d.aki have been trying to get an internship at Symbolab, and they have been given an \textit{at-home assignment} to complete. The assignment is to implement a primitive derivative calculator in Python from scratch. Help them out by:
  \begin{itemize}
    \item Designing classes that adhere to the \textit{Class Design Recipe} that represent important elements of a derivative.
    \item Using \textit{Polymorphism and Inheritance} to represent the different types of derivatives.
    \item Using \textit{Trees and Recursion} to represent the structure of the derivative.
    \item Implementing a \texttt{derivative} method that takes a function and returns its derivative.
  \end{itemize}
\end{frame}

\begin{frame}
  \frametitle{Conclusion}
\begin{center}
  \textbf{Thank you for attending the CSC148 Mega FSG!}
  \end{center}
  \begin{itemize}
    \item We hope you enjoyed the FSG!
    \item We hope you learned something new!
    \item We hope you're ready for the exam!
  \end{itemize}
  \textbf{Thank you for your continued support and participation!}
\end{frame}

\end{document}