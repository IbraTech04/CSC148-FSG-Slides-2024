\documentclass[hyperref={colorlinks,citecolor=blue,linkcolor=blue,urlcolor=blue}, aspectratio=1610]{beamer}
\usepackage{xcolor}
\usepackage{pgfpages}
\usepackage[utf8]{inputenc}
\usepackage[english]{babel}
\usepackage{amsmath}
\usepackage{amsmath,amssymb}
\usepackage{tikz}
\usepackage{forest}
\usepackage{qrcode}
\usepackage{graphicx}


\mode<presentation>
{
  \usetheme{Madrid}       % or try default, Darmstadt, Warsaw, ...
  \usecolortheme{beaver} % or try albatross, beaver, crane, ...
  \usefonttheme{structurebold}    % or try default, structurebold, ...
  \setbeamertemplate{navigation symbols}{}
  \setbeamertemplate{caption}[numbered]
} 



\definecolor{codegreen}{rgb}{0,0.6,0}
\definecolor{codegray}{rgb}{0.5,0.5,0.5}
\definecolor{codepurple}{rgb}{0.58,0,0.82}
\definecolor{backcolour}{rgb}{0.95,0.95,0.92}

\usepackage{listings}
\lstdefinestyle{mystyle}{
    commentstyle=\color{codegreen},
    keywordstyle=\color{blue},
    stringstyle=\color{codepurple},
    basicstyle=\ttfamily\small,
    breakatwhitespace=false,
    breaklines=true,
    captionpos=b,
    keepspaces=true,
    showspaces=false,
    showstringspaces=false,
    showtabs=false,
    tabsize=2
}

% \pgfpagesuselayout{resize to}[%
%   physical paper width=8in, physical paper height=6in]

\title[IbraFSG 6: Midterm 2]{IbraFSG\texttrademark{} 6 - Week 10; Expression Trees and Midterm Review}


\author{Ibrahim Chehab}
\institute{UTM RGASC}
\date{\today}

\begin{document}

\begin{frame}
  \titlepage
\end{frame}

\begin{frame}{Table of Contents}
  \tableofcontents
\end{frame}

\section{Introduction}

\subsection{FSG HouseKeeping}
\begin{frame}
  \frametitle{FSG HouseKeeping}
  \begin{enumerate}
    \item \textbf{Reminder that a new FSG Session has been added:}
    \begin{itemize}
      \item \textbf{When?} \textit{Fridays, 12:00-1:00PM}
      \item \textbf{Where?} \textit{IB210}
    \end{itemize}
    \pause
    \item \textbf{Assignment 2 Deadline Coming Up!} Assignment 2 is due on April 3rd! Make sure you're actively working on it!
    \pause
    \item \textbf{New Study Session Drop Ins!} \\ Feeling Stressed about exams? Don't know where to start? Drop by the MN3260 for a study skills session! 
    \begin{itemize}
      \item \textbf{When:} March $20^{th}$ 10-11AM, March $27^{th}$ 3-4PM, and April $3^{rd}$ 1-2PM
      \item \textbf{Where:} MN3260
    \end{itemize}
    \pause
    \item \textbf{Join the UTM CS Discord Server!} \url{https://discord.gg/utmcs}
  \end{enumerate}

\end{frame}

\subsection{Welcome back to IbraFSGs\texttrademark{}}
\begin{frame}
  \frametitle{Welcome back to IbraFSGs\texttrademark{}}
  \begin{itemize}
  \item Welcome back to IbraFSGs\texttrademark{}! Hello to new people and welcome back to tenured members.
  \item This week we will be quickly going over expression trees, and then going over content for your midterms
  \item There are several use cases for expression trees:
  \begin{itemize}
    \item Evaluating expressions (duh...)
    \item Integral/Derivative calclators (Symbolab and Wolfram Alpha are just big expression trees)
    \item Rendering out a canvas/timeline in creative applications (Photoshop, Premiere, etc.)
    \item Engineering simulations
    \begin{itemize}
      \item Electrical Circuit Analysis
      \item Fluid Dynamics
      \item Computational Chemistry and Physics
      \item Structural Analysis
    \end{itemize}
    \item Compiling and optimizing source code
  \end{itemize}
  \pause
  \item \textbf{Note:} Memorize at least one of the above use cases for the midterm - Oftentimes a question is \textit{What is a use case for expression trees?}

\end{itemize}
\end{frame}

\subsection{A Recap of the UltraSheet\texttrademark{}}
\begin{frame}{A Recap of the UltraSheet\texttrademark{}}
  \begin{itemize}
    \item An \textit{UltraSheet\texttrademark{}} is a "cheat sheet" that you compile for \textbf{yourself} to review course materials 
    \begin{itemize}
      \item Sharing UltraSheets\texttrademark{} is \textbf{counter-productive} and \textbf{will not help you learn the material}
      \item However, reviewing content in a group and simultaneously updating your UltraSheets\texttrademark{} is a good idea
    \end{itemize}
    \item It acts like your own personalized textbook chapter
    \begin{itemize}
      \item It allows you to \textbf{regurgitate all the course information in a contiguous, organized manner} and helps you \textbf{find gaps in your knowledge}
      \item You should \textbf{not} be copying the textbook or lecture slides verbatim; You should be \textbf{summarizing} the content in your own words while tying in examples and analogies
    \end{itemize}
    \item UltraSheets\texttrademark{} help with type 1 and 2 questions 
  \item With your midterm coming up, you should have at least one UltraSheet\texttrademark{} for each week of content, or at least the content you're struggling with
  \end{itemize}

\end{frame}

\section{Expression Trees}

\subsection{Key Terms}
\begin{frame}{Key Terms}
  \textbf{Required Key Terms:} The following key terms are required for this week's content. You should be able to define and explain these terms in your UltraSheets\texttrademark{}:
  \begin{itemize}
    \item \textbf{Expression}
    \item \textbf{Expression Tree}
    \item \textbf{In-Order Traversal}
    \begin{itemize}
      \item \textbf{Note:} Not all expression trees are binary; However most expression trees in this course will be binary
    \end{itemize}
  \item \textbf{Polymorphism}
  \begin{itemize}
    \item \textbf{Why?} Expression Trees abuse Polymorphism to exist
  \end{itemize}
  \end{itemize}
\end{frame}

% Slide omitted - will be kept for next week

% \begin{frame}{Key Terms (Cont'd)}
%     \textbf{Suggested Key Terms:} The following key terms are \textit{recommended} for further studying this week's content. Most are out of the scope of the course and are \textbf{not} required. We don't have time to cover these today, we will be focusing on the midterm review.

%     \begin{itemize}
%       \item \textbf{Big-O Notation ($\mathcal{O}(n)$)}
%       \begin{itemize}
%         \item Upper Bound
%         \item A function $f(n)$ is $\mathcal{O}(g(n))$ if there exists a constant $c > 0$ and $n_0 > 0$ such that $f(n) \leq c \cdot g(n)$ for all $n \geq n_0$
%       \end{itemize}
%       \item \textbf{Big-Omega Notation ($\Omega(n)$)}
%       \begin{itemize}
%         \item Lower Bound
%         \item A function $f(n)$ is $\Omega(g(n))$ if there exists a constant $c > 0$ and $n_0 > 0$ such that $f(n) \geq c \cdot g(n)$ for all $n \geq n_0$
%       \end{itemize}
%       \item \textbf{Big-Theta Notation ($\Theta(n)$)}
%       \begin{itemize}
%         \item Tight Bound
%         \item A function $f(n)$ is $\Theta(g(n))$ if and only if $f(n)$ is $\mathcal{O}(g(n))$ and $f(n)$ is $\Omega(g(n))$
%         \item In other words; $f(n)$ is $\Theta(g(n))$ if and only if there exist constants $c_1 > 0$, $c_2 > 0$, and $n_0 > 0$ such that $c_1 \cdot g(n) \leq f(n) \leq c_2 \cdot g(n)$ for all $n \geq n_0$
%       \end{itemize}
%     \end{itemize}
%     \textbf{Note:} These are CSC236 concepts, and are \textbf{not} required for CSC148. I am including these to complement your lecture on efficiency and complexity this week.
% \end{frame}

\section{Midterm Tips and Tricks}
\subsection{Key Concepts}

\begin{frame}
  \frametitle{Key Concepts - My Thoughts} 
  The ultimate take-aways from Weeks 5-10 are as follows:
  \begin{enumerate}
    \item \textbf{Recursion}
    \pause
    \item \textbf{List Comprehensions}
    \pause
    \item \textbf{Trees}
    \pause
    \item \textbf{Efficiency and Complexity}
  \end{enumerate}
\end{frame}

\begin{frame}
  \frametitle{Key Concepts - My Thoughts (Cont'd)} 
  The ultimate take-aways from Weeks 5-10 are as follows:
  \begin{enumerate}
    \item \textbf{Recursion}
    \begin{itemize}
      \item Base Case
      \item Recursive Case
      \item My Recursion Analogy
    \end{itemize}
    \item \textbf{List Comprehensions}
    \begin{itemize}
      \item Syntax
      \item Use Cases
      \item Chaining/Composition
    \end{itemize}
    \item \textbf{Trees}
    \begin{itemize}
      \item Traversals
      \item BSTs vs BTs
      \item Efficiency and Complexity
    \end{itemize}
    \item \textbf{Efficiency and Complexity}
    \begin{itemize}
      \item How do we measure efficiency?
      \item Big-O Notation
      \item Recursive Efficiency
    \end{itemize}
  \end{enumerate}
\end{frame}

\subsection{Type 1, 2, and 3 Questions}
\begin{frame}
  \frametitle{Type 1, 2, and 3 Questions}
  You've often heard me refer to \textit{Type 1, 2, and 3} questions. What do I mean by this? \\ \pause
  \begin{itemize}
    \item \textbf{Type 1:} \textit{Theory-based questions}
    \begin{itemize}
      \item These questions are typically pulled straight out of notes/lecture slides
      \item These questions are typically multiple choice or short answer
      \item Study for them using your UltraSheets\texttrademark{}
    \end{itemize}
    \pause 
    \item \textbf{Type 2:} \textit{Knowledge-based questions}
    \begin{itemize}
      \item These questions typically build off of Type 1 questions, and usually require you to have a strong understanding of the material. They require you to reflect on what you learned in lecture + preps
      \item These questions are typically short answer or long answer
      \item Study for them using your UltraSheets\texttrademark{}
    \end{itemize}
    \pause
    \item \textbf{Type 3:} \textit{Application-based questions}
    \begin{itemize}
      \item These questions typically require you to apply your knowledge to a new problem
      \item These questions are typically long answer
      \item Study for them by practicing problems
    \end{itemize}
  \end{itemize}
\end{frame}


\subsection{Midterm-taking Strategies}
\begin{frame}
  \frametitle[Midterm-taking Strategies]{Midterm-taking Strategies}
  \begin{enumerate}
    \item \textbf{Preflight Analysis:}
    \begin{itemize}
      \item Quckly skim through the entire exam to see what you're up against. Make mental notes of the questions you know you can answer, and the questions you're unsure about.
    \end{itemize}
    \pause
    \item \textbf{Butterfly Method:}
    \begin{itemize}
      \item If you get stuck on a question, go to the next question you can solve (see above). This will help you get into a rhythm and build confidence.
      \item Sometimes swapping between questions rapidly in succession can help you solve the question you were stuck on.
    \end{itemize}
    \pause
    \item \textbf{Time Management:}
    \begin{itemize}
      \item Allocate time for each question based on the number of marks it's worth. If you're stuck on a question, move on and come back to it later.
      \item If you're stuck on a question that isn't worth much, skip it and come back to it later. Chances are you're overthinking it.
      \item Even if you don't end up figuring it out, if it was only worth 1-2 marks, it's not the end of the world. Prioritize the questions worth more marks.
      \item A question worth more marks $\implies$ it's harder
    \end{itemize}
  \end{enumerate}
\end{frame}

\section{Practice Problems}

\begin{frame}
  \frametitle{Jamboard Link}
  \begin{center}
    \qrcode[height=2in]{https://jamboard.google.com/d/1yhpEvaeEAau7L-e_-YdVWuTNVfsFthu7QRAheAsgm_Y/edit?usp=sharing}\\
    \url{https://tinyurl.com/ibrafsg0320}\\
    \textit{0320 for March 20th}\\
    \textbf{Note:} Don't fool around with the Jamboard, it's for your benefit.
  \end{center}
\end{frame}

\subsection{Practice Problem I: TreeTheory Pro Max}

\begin{frame}[fragile]
  
  \frametitle{Practice Problem 1.1}
  Which of the following statements is true? Select all that apply:  
  \begin{enumerate}
    \item The in-order traversal of a binary tree is always sorted
    \item The post-order traversal of a binary search tree \textit{might} be sorted
    \item The pre-order traversal of a binary tree \textit{might} be sorted
    \item The pre-order traversal of a binary search tree is always sorted
    \item \textbf{All} ancestors of a node in a binary search tree are less than the node
    \item \textbf{All} ancesors of a node in a binary tree \textit{might} be less than the node
  \end{enumerate}
  \onslide<2->{
    \textbf{Pay close attention to the usage of \textit{Binary Search Tree} and \textit{Binary Tree} in questions like these}\\
    Obviously, we know they are \textbf{NOT} interchangeable, but the question might be trying to trick you.
  }
\end{frame}

\begin{frame}
  \frametitle{Practice Problem 1.2}
  Given the in-order traversal of an arbitrary binary \textit{search} tree, is it possible to reconstruct the original tree? If so, how? If not, why not?
  
  \textit{Hint: Recall our discussion on AVL Trees from last week}\\
  \textit{Hint 2: Recall your lab on BST rotations}

  \onslide<2->{
    Consider the two following trees:
    \begin{center}
      \begin{forest}
        [5
          [3
            [2]
            [4]
          ]
          [7
            [6]
            [8]
          ]
        ]
      \end{forest}
      \hspace{1cm}
      \begin{forest}
        [5
          [2
          [,phantom]
          [3
            [,phantom]  
            [4]
              ]
          ]
          [7
            [8]
            [6]
          ]
        ]
      \end{forest}
    \end{center}
  }
\end{frame}

\begin{frame}[fragile]
  \frametitle{Practice Problem 1.3}
  Given a sorted list of integers from $1$ to $n, n \in \mathbb{N} \setminus \{0\}$, how do you construct a balanced binary search tree (i.e: height $\log(n)$) from this list? Outline the order of insertion, and explain why this works.\\
  \onslide<2->{
    Hint: Consider what is required for a BST to be balanced
  }

\end{frame}

\subsection{Practice Problem II: Itr2Recur}
\begin{frame}[fragile]
  \frametitle{Practice Problem II: Itr2Recur}
  Convert the following iterative function to a recursive function:

  \begin{lstlisting}[language=Python, style=mystyle]
def ibranatchi_iterative(n: int) -> int:
  if n == 0:
    return 0
  elif n == 1:
    return 1
  elif n == 2:
    return 5

  sequence = [0, 1, 5]
  for i in range(3, n + 1):
    next_value = sequence[i - 1] * 2 * sequence[i - 2] - 5 * sequence[i - 3]
    sequence.append(next_value)

  return sequence[-1]
 
def ibranatchi_recursive(n: int) -> int:
  # TODO: Implement this recursively:

  \end{lstlisting}
\end{frame}

\subsection{Practice Problem III: List Comprehensions}
\begin{frame}[fragile]
  \frametitle{Practice Problem III: List Comprehensions}
  What is the output of the following list comprehension. If it throws an error, explain why:
  \begin{lstlisting}[language=Python,style=mystyle]
    [[x*y for x in range(4)] for y in range(7)]
  \end{lstlisting}
  
  what if we switch the \texttt{x} and \texttt{y} iterables?

\end{frame}

\section{Conclusion}
\begin{frame}
  \frametitle{Thank you for coming!}
  \centering
  \begin{figure}
    \includegraphics[width=0.5\textwidth]{that's_all_folks.jpeg}
    \caption{Image courtsey of Looney Tunes\texttrademark{} and Warner Bros.\texttrademark{}}
  \end{figure}

  \textbf{Good luck, everyone!}
\end{frame}

\end{document}